\documentclass{article}

\usepackage[english]{babel}
\usepackage{ragged2e}
\usepackage{hyperref}
\hypersetup{
    colorlinks=true,
    linkcolor=blue,
    filecolor=magenta,      
    urlcolor=cyan,
    }

\usepackage{listings}
\usepackage{xcolor}

\definecolor{codegreen}{rgb}{0,0.6,0}
\definecolor{codegray}{rgb}{0.5,0.5,0.5}
\definecolor{codepurple}{rgb}{0.58,0,0.82}
\definecolor{backcolour}{rgb}{0.95,0.95,0.92}

\lstdefinestyle{mystyle}{
    backgroundcolor=\color{backcolour},   
    commentstyle=\color{codegreen},
    keywordstyle=\color{magenta},
    numberstyle=\tiny\color{codegray},
    stringstyle=\color{codepurple},
    basicstyle=\ttfamily\footnotesize,
    breakatwhitespace=false,         
    breaklines=true,                 
    captionpos=b,                    
    keepspaces=true,                 
    numbers=left,                    
    numbersep=5pt,                  
    showspaces=false,                
    showstringspaces=false,
    showtabs=false,                  
    tabsize=2
}

\lstset{style=mystyle}


\usepackage[letterpaper,top=2cm,bottom=2cm,left=3cm,right=3cm,marginparwidth=1.75cm]{geometry}

\usepackage{amsmath}
\usepackage{graphicx}

\title{Trabalho - Projeto de BD}
\title{Projeto de BD - UBER, fase III}

\author{Guilherme Moreira Castilho - B51987,\\ Maria Eduarda Mesquita Magalhães - B51085,\\ Vitor Matheus do Nascimento Moreira - B51092}
\date{Novembro de 2023}

\begin{document}
\maketitle


\section*{\textbf{Banco de Dados Uber}}

\section{Sobre}

A presente etapa do do trabalho trabalho envolve a implementação dos dados gerados nas etapas anteriores no SGBD no postgresql, além de carregar os registros já gerados, vamos também apresentar algumas instruções SQL úteis em nosso projeto.

\section{Instruções DML e DDL}

Nessa etapa, criamos as nossas tabelas e convertemos os dados que havíamos criados no Excel para inserção via SQL.
\begin{enumerate}
    \item instruções DDL: \href{https://github.com/eduardammag/A2-BD-2023.2/blob/main/DDl%20create%20tables%20UBER.sql}{DLL create tables UBER}
    \item instruções DDL: \href{https://github.com/eduardammag/A2-BD-2023.2/blob/main/DML%20insert%20table%20UBER.sql}{DML insert tables UBER}
\end{enumerate}

\subsection{Desafios}

Nessa etapa, um de nossos desafios foi durante a criação das tabelas, criar cada coluna com o tipo correto para posterior inserção de dados demanda um trabalho cuidadoso, e foi necessário refazer algumas colunas pois os tipos não batiam com os necessários. Inicialmente estávamos criando todas tabelas para depois inserir os dados, mas notamos que o trabalho de criação de tabelas e inserção de dados deveria ser algo em conjunto, e mudamos nossa metodologia para criar as tabelas/colunas e em seguida inserir os respectivos registros. Por fim, juntamos todas instruções nos arquivos disponiveis acima.

\section{Comandos SQL úteis}

Nessa etapa, criamos alguns comandos SQL que podem ser úteis para nosso tema.

\subsection{Listando o preço e os nomes do clientes e motorista de um pedido}

\begin{lstlisting}[language=SQL]
SELECT 
    cliente.nome AS "nome cliente",
    motorista.nome AS "nome motorista",
    pedido.preco AS "preco pedido"
FROM pedido
INNER JOIN cliente ON pedido.idcliente = cliente.idcliente
INNER JOIN motorista ON pedido.idmotorista = motorista.idmotorista
ORDER BY pedido.preco DESC;
\end{lstlisting}


\begin{table}[h]
\centering
\begin{tabular}{l|l|r}
    
\textbf{nome cliente} & \textbf{nome motorista} & \textbf{preco pedido} \\
\hline
 Gabriela Lopes Moreira     & Till Lindemann        &       \$66.08 \\
 João Vitor Manuel Sousa    & Yago Marreta Lopes    &       \$57.30 \\
 Guilherme Juliano Castilho & Felipe Juan Pereira   &       \$45.31 \\
 Eduarda Valentim Moura     & Marilia Rezende       &       \$37.49 \\
 Gabriel Santos Biancardi   & Isaac Brock           &       \$23.80 \\
 Joana Mesquita Magalhães   & Felipe Juan Pereira   &       \$16.97 \\
 Luisa Sonza Quadros        & Chaves Teixeira       &       \$14.50 \\
 Milena Justino Tavares     & Adriana Costeleta     &       \$11.23 \\
 Natalia Fontenelle Araujo  & Lavínia Emilly Trovão &       \$9.74 \\
 Ruana Oliveira Fernandes   & Thiago Manoel Lima    &       \$5.50 

\end{tabular}
\end{table}


No comando SQL acima, fazemos uma consulta que envolve três tabelas, "pedido", "cliente" e "motorista", usando INNER JOIN para combinar registros da tabela cliente na tabela pedido com base nos ids dos clientes, e outro INNER JOIN para combinar registros da tabela motorista na tabela pedido com base nos ids dos motoristas. Como selecionamos as colunas "nome" da tabela "cliente", a coluna "nome" da tabela "motorista" e a coluna "preco" da tabela "pedido", a consulta retorna uma tabela de nomes de clientes, nomes de motoristas e os preços de seus pedidos correspondentes, ordenados do maior para o menor preço. A tabela "pedido" é usada como a tabela principal, e as tabelas "cliente" e "motorista são usadas para obter informações adicionais sobre os clientes e motoristas associados ao pedido. Acreditamos que esse comando é útil pois podemos querer saber quais pessoas estão envolvidas nos pedidos de maior ou menos valor.

\subsection{Mostrando um ranking de clientes conforme seus saldos}

\begin{lstlisting}[language=SQL]
SELECT
    nome AS "nome cliente",
    saldo,
    RANK() OVER (ORDER BY saldo DESC) AS ranking
FROM cliente
ORDER BY saldo DESC;
\end{lstlisting}


\begin{table}[h]
    \centering
    \begin{tabular}{l|r|r}
    
    \textbf{nome cliente } & \textbf{saldo} & \textbf{ranking} \\
    \hline

    João Vitor Manuel Sousa     &   1500.02     &        1  \\
    Ruana Oliveira Fernandes    &    600.00     &        2  \\
    Gabriel Santos Biancardi    &    562.33     &        3  \\
    Milena Justino Tavares      &    231.47     &        4  \\
    Guilherme Juliano Castilho  &    230.15     &        5  \\
    Gabriela Lopes Moreira      &    170.90     &        6  \\
    Luisa Sonza Quadros         &     57.99     &        7  \\
    Eduarda Valentim Moura      &     18.78     &        8  \\
    Natalia Fontenelle Araujo   &     16.89     &        9  \\
    Joana Mesquita Magalhães    &     10.32     &       10
    
\end{tabular}
\end{table}

Este comando faz um ranking de clientes conforme seus saldos no aplicativo. Primeiro selecionamos as colunas úteis da tabela cliente, e criamos uma nova coluna chamada "ranking" a partir do ranking dos saldos dos clientes, depois exibimos a tabela de forma descrescente em relação ao saldo. Esse comando pode ser útil em casos onde se quer saber quais clientes tem o maior valor em saldo, para distribuição de descontos por exemplo (clientes com saldo maior gastam mais, portanto podem receber mais descontos).

\break

\subsection{Mostrando o valor total transacionado por Estado}

\begin{lstlisting}[language=SQL]
SELECT 
    localidade.estado, 
    SUM(pedido.preco) AS "valor total transacionado"
FROM viagem
JOIN pedido ON viagem.idpedido = pedido.idpedido
JOIN localidade ON localidade.idlocal = viagem.idlocaldestino
GROUP BY localidade.estado
ORDER BY "valor total transacionado" DESC;
\end{lstlisting}

\begin{table}[h]
    \centering
    \begin{tabular}{l|r}
    
    \textbf{estado} & \textbf{valor total transacionado} \\
    \hline

    Minas Gerais    &                   \$145.60  \\
    São Paulo       &                   \$99.62   \\
    Rio de Janeiro  &                   \$31.47   \\
    Parana          &                   \$11.23 
    
\end{tabular}
\end{table}

No comando acima, listamos o valor total transacionado por estado de acordo com o local de destino, isto é, quais estados de destino transacionam o maior valor. Primeiro selecionamos a coluna estado da tabela "localidade", depois criamos uma coluna que ira somar os preços das viagens. Fazemos essa seleção na tabela viagem, e depois juntamos a tabela pedido conforme o id do pedido, e juntamos a localidade conforme o idlocal da tabela "localidade" e o "idlocaldestino" da tabela "viagem", por fim agrupamos nossos registros conforme o estado, e exibimos as colunas selecionadas coforme orda descrescente do valor total transacionado. Esse comando pode ser útil em uma situação na qual se quer saber quais estados atraem mais clientes, para um possivel investimento por parte da UBER, por exemplo.

\subsection{Mostrando o valor total rotacionado por cada motorista}
\begin{lstlisting}[language=SQL]
SELECT
    motorista.nome AS "nome motorista",
    SUM(pedido.preco) AS "valor rotacionado total"
FROM pedido
INNER JOIN motorista ON pedido.idmotorista = motorista.idmotorista
GROUP BY motorista.idmotorista
ORDER BY "valor rotacionado total" DESC;
\end{lstlisting}


\begin{table}[h]
\centering
\begin{tabular}{l|r}

\textbf{Nome Motorista} & \textbf{Valor Rotacionado Total} \\
\hline

Till Lindemann          &   \$66.08 \\
Felipe Juan Pereira     &   \$62.28 \\
Yago Marreta Lopes      &   \$57.30 \\
Marilia Rezende         &   \$37.49 \\
Isaac Brock             &   \$23.80 \\
Chaves Teixeira         &   \$14.50 \\
Adriana Costeleta       &   \$11.23 \\
Lavínia Emilly Trovão   &   \$9.74 \\
Thiago Manoel Lima      &   \$5.50 \\

\end{tabular}
\end{table}

Neste comando, verificamos o total rotacionado por cada motorista. Primeiro selecionamos as colunas "nome" da tabela motorista e criamos uma coluna que ira somar os preços, fazemos tal seleção na tabela pedido, e concatenamos as partes de interesse da tabela motorista conforme o id do motorista. Depois agrupamos por motorista (o que faz também a soma da coluna de valor rotacionado), e retornamos a tabela em ordem descrescente de valor. Esse comando pode ser interessante quando se quer equilibrar o valor rotacionado por cada motorista por exemplo, (sugerir mais corridas para os que estão rodando pouco valor).

\end{document}


